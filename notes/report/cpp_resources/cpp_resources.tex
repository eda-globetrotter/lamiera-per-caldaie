%	This is written by Zhiyang Ong to document usage of C++ STL features that I can use for my C++ -based software, particularly those in electronic design automation.

%	The MIT License (MIT)

%	Copyright (c) <2014> <Zhiyang Ong>

%	Permission is hereby granted, free of charge, to any person obtaining a copy of this software and associated documentation files (the "Software"), to deal in the Software without restriction, including without limitation the rights to use, copy, modify, merge, publish, distribute, sublicense, and/or sell copies of the Software, and to permit persons to whom the Software is furnished to do so, subject to the following conditions:

%	The above copyright notice and this permission notice shall be included in all copies or substantial portions of the Software.

%	THE SOFTWARE IS PROVIDED "AS IS", WITHOUT WARRANTY OF ANY KIND, EXPRESS OR IMPLIED, INCLUDING BUT NOT LIMITED TO THE WARRANTIES OF MERCHANTABILITY, FITNESS FOR A PARTICULAR PURPOSE AND NONINFRINGEMENT. IN NO EVENT SHALL THE AUTHORS OR COPYRIGHT HOLDERS BE LIABLE FOR ANY CLAIM, DAMAGES OR OTHER LIABILITY, WHETHER IN AN ACTION OF CONTRACT, TORT OR OTHERWISE, ARISING FROM, OUT OF OR IN CONNECTION WITH THE SOFTWARE OR THE USE OR OTHER DEALINGS IN THE SOFTWARE.

%	Email address: echo "cukj -wb- 23wU4X5M589 TROJANS cqkH wiuz2y 0f Mw Stanford" | awk '{ sub("23wU4X5M589","F.d_c_b. ") sub("Stanford","d0mA1n"); print $5, $2, $8; for (i=1; i<=1; i++) print "6\b"; print $9, $7, $6 }' | sed y/kqcbuHwM62z/gnotrzadqmC/ | tr 'q' ' ' | tr -d [:cntrl:] | tr -d 'ir' | tr y "\n"

%%%%%%%%%%%%%%%%%%%%%%%%%%%%%%%%%%%%%%%%%%%%%%



%%%%%%%%%%%%%%%%%%%%%%%%%%%%%%%%%%%%%%%%%%%
\chapter{C++ Resources}
\label{chp:CppResources}


Some {C++} and C++ STL resources are: \vspace{-0.3cm}
\begin{enumerate} \itemsep -4pt
\item \cite{Mohtashim2015a}: \url{http://www.tutorialspoint.com/cplusplus/cpp_stl_tutorial.htm}
\item \cite{CplusplusCom2014} and \url{CplusplusCom2015}: \url{http://www.cplusplus.com/reference/stl/}
\item \url{http://en.cppreference.com/w/cpp/container}
\item \url{http://www.cs.wustl.edu/~schmidt/PDF/stl4.pdf}
\item Pointers to functions: \url{http://www.cplusplus.com/doc/tutorial/pointers/}
\end{enumerate}


C++ topics: \vspace{-0.3cm}
\begin{enumerate} \itemsep -4pt
\item Function objects: \vspace{-0.3cm}
	\begin{enumerate} \itemsep -2pt
	\item \url{https://en.wikipedia.org/wiki/Functional_(C%2B%2B)}
	\item \url{http://stackoverflow.com/questions/356950/c-functors-and-their-uses}
	\item \url{http://www.cprogramming.com/tutorial/functors-function-objects-in-c++.html}
	\end{enumerate}
\item Strings: \vspace{-0.3cm}
	\begin{enumerate} \itemsep -2pt
	\item \cite{Stroustrup2014}, Chp 23
	\item \cite{Stroustrup2009}, Chp 23
	\item \cite{Gregoire2014}, Chp 18
	\item \cite{Allain2012}, Chp 19
	\item \cite{Eckel2003}, Chp 1
	\end{enumerate}
\item IO Streams: \vspace{-0.3cm}
	\begin{enumerate} \itemsep -2pt
	\item \cite{Eckel2003}, Chp 2
	\item \cite{Gaddis2010}, Chp 12. See all of \cite{Gaddis2010,Gaddis2011,Gaddis2012}.
	\item \cite{Stroustrup2014}, Chp 10-11
	\item \cite{Stroustrup2009}, Chp 10-11
	\item \cite{Oualline2003}, Chp 16
	\item \cite{Vermeir2001}, Chp 10
	\item \cite{Schildt2003}, Chp 21
	\item \cite{Allain2012}, Chp 28
	\item \cite{Gregoire2014}, Chp 12
	\item \cite{Prata2012}, Chp 17
	\item \cite{Lippman2013}, Chp 8
	\end{enumerate}
\item Templates: \vspace{-0.3cm}
	\begin{enumerate} \itemsep -2pt
	\item \cite{Eckel2003}, Chp 3
	\item \cite{Eckel2000}, Chp 16
	\item \cite{Stroustrup2014}, Chp 19
	\item \cite{Stroustrup2009}, Chp 19
	\item \cite{Oualline2003}, Chp 24
	\item \cite{Vermeir2001}, Chp 6
	\item \cite{Alexandrescu2001}, book; typelist - Chp 3
	\item \cite{Schildt2003}, Chp 18
	\item \cite{Vandevoorde2003}, book
	\item \cite{Abrahams2005}, book
	\item \cite{Allain2012}, Chp 29
	\item \cite{Gregoire2014}, Chp 11,21
	\item \cite{Lippman2013}, Chp 16
	\end{enumerate}
\item Debugging: \vspace{-0.3cm}
	\begin{enumerate} \itemsep -2pt
	\item \cite{Eckel2003}, Chp 11 (especially memory management problems, pp. 533)
	\end{enumerate}
\item STL containers: \vspace{-0.3cm}
	\begin{enumerate} \itemsep -2pt
	\item \cite{Eckel2003}, Chp 4
	\item \cite{Schildt2004a}, Chp 8
	\item \cite{Oualline2003}, Chp 25
	\item \cite{Vermeir2001}, Chp 7
	\item \cite{Reese2006a}, book
	\item \cite{Allain2012}, Chp 18
	\item \cite{Gregoire2014}, Chp 15-16
	\item \cite{Prata2012}, Chp 16
	\item \cite{Lippman2013}, Chp 9,11
	\item \cite{EliteHussar2010}: \vspace{-0.2cm}
		\begin{enumerate} \itemsep -2pt
		\item {\tt vector$<$int$>$ v(10);} \hspace{0.2in} //{\it\ Create an int vector of size 10.}
		\item {\tt v[5] = 10;} //{\it\ Target of this assignment is the return value of operator[].}
		\end{enumerate}
	\end{enumerate}
\item STL algorithms: \vspace{-0.3cm}
	\begin{enumerate} \itemsep -2pt
	\item \cite{Eckel2003}, Chp 5
	\item \cite{Oualline2003}, Chp 25
	\item \cite{Vermeir2001}, Chp 7
	\item \cite{Reese2006a}, book
	\item \cite{Allain2012}, Chp 18
	\item \cite{Gregoire2014}, Chp 15,17
	\item \cite{Prata2012}, Chp 16
	\item \cite{Lippman2013}, Chp 10
	\end{enumerate}
\item Function addresses: \vspace{-0.3cm}
	\begin{enumerate} \itemsep -2pt
	\item \cite{Eckel2000}, Chp 3, pp. 213
	\item \cite{Stroustrup2014}, Chp 8
	\item \cite{Stroustrup2009}, Chp 8
	\end{enumerate}
\item Dynamic memory management problems: \vspace{-0.3cm}
	\begin{enumerate} \itemsep -2pt
	\item \cite{Eckel2000}, Chp 6,13
	\item \cite{Gaddis2010}, Chp 13. See all of \cite{Gaddis2010,Gaddis2011,Gaddis2012}.
	\item \cite{Meyer2005}, Chp 2-4
	\item \cite{Schildt2003}, Chp 29
	\item \cite{Allain2012}, Chp 14
	\item \cite{Gregoire2014}, Chp 10,22
	\item \cite{Prata2012}, Chp 9,12
	\item \cite{Lippman2013}, Chp 12,13
	\end{enumerate}
\item Function overloading: \vspace{-0.3cm}
	\begin{enumerate} \itemsep -2pt
	\item \cite{Eckel2000}, Chp 7
	\item \cite{Gaddis2010}, Chp 6. See all of \cite{Gaddis2010,Gaddis2011,Gaddis2012}.
	\item \cite{Stroustrup2014}, Chp 8
	\item \cite{Stroustrup2009}, Chp 8
	\item \cite{Schildt2003}, Chp 14
	\end{enumerate}
\item Operator overloading: \vspace{-0.3cm}
	\begin{enumerate} \itemsep -2pt
	\item \cite{Eckel2000}, Chp 12
	\item \cite{Oualline2003}, Chp 18
	\item \cite{Schildt2003}, Chp 15
	\item \cite{Lippman2013}, Chp 14
	\end{enumerate}
\item Constants: \vspace{-0.3cm}
	\begin{enumerate} \itemsep -2pt
	\item \cite{Eckel2000}, Chp 8
	\end{enumerate}
\item Functions and pointers: \vspace{-0.3cm}
	\begin{enumerate} \itemsep -2pt
	\item \cite{Eckel2000}, Chp 11: \vspace{-0.2cm}
		\begin{enumerate} \itemsep -2pt
		\item use const at the end of accessor functions
		\item Do not use pointers as instance variables
		\end{enumerate}
	\item \cite{Stroustrup2014}, Chp 8: \vspace{-0.2cm}
		\begin{enumerate} \itemsep -2pt
		\item Pass-by-reference: e.g., void init(vector$<$double$>$ \&v)
		\item Pass-by-const-reference: e.g., void print(const vector$<$double$>$ \&v)
		\item Pass-by-value: e.g., void fn(int x)
		\end{enumerate}
	\item \cite{Stroustrup2009}, Chp 8
	\item \cite{Oualline2003}, Chp 15,20
	\item \cite{Allain2012}, Chp 12-13
	\item \cite{Prata2012}, Chp 7-8
	\item \cite{Lippman2013}, Chp 6
	\item Elsewhere: \vspace{-0.2cm}
		\begin{enumerate} \itemsep -2pt
		\item You cannot call a non-const method from a const method. That would 'discard' the const qualifier.: \vspace{-0.1cm}
			\begin{enumerate} \itemsep -1pt
			\item \url{http://stackoverflow.com/questions/2382834/discards-qualifiers-error}
			\end{enumerate}
		\item Pointer to constant data: {\it const type$^{\ast}$ variable;} and {\it type const $^{\ast}$ variable;} \vspace{-0.1cm}
			\begin{enumerate} \itemsep -1pt
			\item \url{http://www.cprogramming.com/reference/pointers/const_pointers.html}
			\end{enumerate}
		\item Pointer with constant memory address: {\it type $^{\ast}$ const variable = some-memory-address;} \vspace{-0.1cm}
			\begin{enumerate} \itemsep -1pt
			\item \url{http://www.cprogramming.com/reference/pointers/const_pointers.html}
			\end{enumerate}
		\item Constant data with a constant pointer: {\it const type $^{\ast}$ const variable = some-memory-address;} and {\it type const $^{\ast}$ const variable = some-memory-address;} \vspace{-0.1cm}
			\begin{enumerate} \itemsep -1pt
			\item \url{http://www.cprogramming.com/reference/pointers/const_pointers.html}
			\end{enumerate}
		\item \url{http://stackoverflow.com/questions/1143262/what-is-the-difference-between-const-int-const-int-const-and-int-const} \cite{Mortensen2015}: \vspace{-0.1cm}
			\begin{enumerate} \itemsep -1pt
			\item Read it backwards; the first {\it const} can be on either side of the type.
			\item ``Read pointer declarations right-to-left.''
			\item From the answer of Ted Dennison, July 17, 2009. {\bf Rule: The ``const'' goes after the thing it applies to. Putting const at the very front (e.g., const int $^{\ast}$) is an exception to the rule.}
			\item int$^{\ast}$ -- pointer to int
			\item int const $^{\ast}$ == const int $^{\ast}$ -- pointer to const int
			\item int $^{\ast}$ const -- const pointer to int
			\item int const $^{\ast}$ const == const int $^{\ast}$ const -- const pointer to const int
			\item int $^{\ast}$$^{\ast}$ -- pointer to pointer to int
			\item int $^{\ast}$$^{\ast}$ const -- A const pointer to a pointer to an int
			\item int $^{\ast}$ const $^{\ast}$ -- A pointer to a const pointer to an int
			\item int const $^{\ast}$$^{\ast}$ -- A pointer to a pointer to a const int
			\item int $^{\ast}$ const $^{\ast}$ const -- A const pointer to a const pointer to an int
			\end{enumerate}
		\item For the following \cite{Mortensen2015}, let: {\it int var0 = 0;} \vspace{-0.1cm}
			\begin{enumerate} \itemsep -1pt
			\item {\it const int {\rm \&}ptr1 = var0;} // Constant reference
			\item {\it int $^{\ast}$ const ptr2 = {\rm \&}var0;} // Constant pointer
			\item {\it int const $^{\ast}$ ptr3 = {\rm \&}var0;} // Pointer to const
			\item {\it const int $^{\ast}$ const ptr4 = {\rm \&}var0;} // Const pointer to a const
			\end{enumerate}
		\item A pointer is dereferenced via the explicit $^{\ast}$ operator. The $^{\ast}$ operator should not be used to dereference a reference (variable) \cite{Saks2001}.
		\item \cite{Saks2001}: \vspace{-0.1cm}
			\begin{enumerate} \itemsep -1pt
			\item int $^{\ast}$pi = {\&}i; // Indirect expression to dereference $pi$ to $i$. ``Declare $pi$ as an object of type `pointer to int' whose initial value is the address of object $i$'' \cite{Saks2001a}.
			\item int {\&}ri = i; // $ri$ is dereferenced to refer to $i$. ``Declares $ri$ as an object of type `reference to int' referring to $i$'' \cite{Saks2001a}.
			\item The {\it C++} standard does not dictate how compilers shall implement references. However, popular compilers tend to implement references as pointers. Therefore, there are no significant advantages of using references or pointers.
			\end{enumerate}
		\item \cite{Saks2001a}: \vspace{-0.1cm}
			\begin{enumerate} \itemsep -1pt
			\item ``A valid reference must refer to an object; a pointer need not. A pointer, even a const pointer, can have a null value. A null pointer doesn't point to anything.''
			\item I can bind a reference to a null pointer, but I cannot dereference a null pointer since it can ``produce undefined behavior''.
			\end{enumerate}
		\item \cite{Ozcan2013}: \vspace{-0.1cm}
			\begin{enumerate} \itemsep -1pt
			\item ``A reference is a variable that refers to something else and can be used as an alias for that something else. A pointer is a variable that stores a memory address, for the purpose of acting as an alias to what is stored at that address. So, a pointer is a reference, but a reference is not necessarily a pointer. Pointers are a particular implementation of the concept of a reference, and the term tends to be used only for languages that give you direct access to the memory address. References can be implemented internally in a language using pointers, or using some other mechanism.'' Answer from dan1111.
			\item ``Passing an object by value means making a copy of it. You can modify that copy without affecting the original. Making that copy can cost a lot of memory access though. Passing an object by reference means passing a handle to that object. This is cheaper because you don't need to make a copy. It also means that any changes you make will affect the original.'' Answer from Steve Rowe.
			\item ``There is no such thing as a null reference. A reference must always refer to some object. As a result, if you have a variable whose purpose is to refer to another object, but it is possible that there might not be an object to refer to, you should make the variable a pointer, because then you can set it to null. On the other hand, if the variable must always refer to an object, i.e., if your design does not allow for the possibility that the variable is null, you should probably make the variable a reference.'' Answer from Harssh S. Shrivastava.
			\end{enumerate}
		\end{enumerate}
	\end{enumerate}
\item OOD and inheritance: \vspace{-0.3cm}
	\begin{enumerate} \itemsep -2pt
	\item \cite{Eckel2000}, Chp 14,15
	\item \cite{Gaddis2010}, Chp 13,14,15. See all of \cite{Gaddis2010,Gaddis2011,Gaddis2012}.
	\item \cite{Stroustrup2014}, Chp 9
	\item \cite{Stroustrup2009}, Chp 9
	\item \cite{Oualline2003}, Chp 13-14,21
	\item \cite{Vermeir2001}, Chp 3-4,8
	\item \cite{Allain2012}, Chp 24-26
	\item \cite{Gregoire2014}, Chp 4-9
	\item \cite{Prata2012}, Chp 10-11,13,14,15
	\item \cite{Lippman2013}, Chp 7,15,18,19
	\end{enumerate}
\item SW engineering issues: \vspace{-0.3cm}
	\begin{enumerate} \itemsep -2pt
	\item \cite{Allain2012}, Chp 21
	\item \cite{Gregoire2014}, Chp 24-26
	\end{enumerate}
\item multi-threading: \vspace{-0.3cm}
	\begin{enumerate} \itemsep -2pt
	\item \cite{Schildt2004a}, Chp 3
	\end{enumerate}
\item graphs: \vspace{-0.3cm}
	\begin{enumerate} \itemsep -2pt
	\item \cite{Schildt2004a}, Chp 7
	\end{enumerate}
\item typedef: \vspace{-0.3cm}
	\begin{enumerate} \itemsep -2pt
	\item In the sandbox, use the {\it Make} target {\it make typedef} to study an example of how {\it typedef} can be used. When the {\it header file} defines/specifies the {\it typedef}, and is included in the {\it C++ implementation file} and other {\it C++ implementation file}s that instantiates those objects, it can be used subsequently without additional definition/specification. October 6, 2015.
	\end{enumerate}
\end{enumerate}


Books to classify: \vspace{-0.3cm}
\begin{enumerate} \itemsep -4pt
\item C++ programming: \cite{Horstmann2012,Katupitiya2006,Koenig2000,Pozrikidis2007,Prata2005,Romanik2003,Savitch2009,Scheinerman2006,Schildt1998a,Schildt2003a}
\item C++ STL: \cite{Josuttis2012,Karlsson2006a,Robson2000,HewlettPackardCompanyStaff2014,HewlettPackardCompanyStaff1994,Riesbeck2009,Cline2000,Cline2003,Cline2011}
\item C++ -based MPI programming: \cite{Karniadakis2003}
\item scientific computing: \cite{PittFrancis2012}
\item Boost C++: \cite{Mukherjee2015,Polukhin2013,Schaling2012}
\end{enumerate}


%%%%%%%%%%%%%%%%%%%%%%%%%%%%%%%%%%%%%%%%%%%
\section{Computational Complexity of C++ Containers}
\label{sec:ComputationalComplexityofCppContainers}


Table \ref{tab:ComputationalComplexityofCppContainers} shows a tabulated summary of containers in the {\it C++} Standard Template Library (STL) and the computational complexity for each of their common operations: add(element e), remove(element e), search(element e), size(), empty(), begin(), and end(). \\

%\begin{table}[htdp]
\begin{table}[htp]
\caption{Computational Complexity of Basic Operations of Containers from the {\it C++ STL}.}	\vspace{-0.2in}
\label{tab:ComputationalComplexityofCppContainers}
	\begin{center}
		\begin{tabular}{|c|c|c|c|c|c|c|c|}
		\hline
		Container $\backslash$ Complexity & add & remove & search & size & empty & begin & end \\
		\hline
		vector & O(1) & O(n) & O(n) & O(1) & O(1) & O(1) & O(1) \\
		\hline
		list & O(1) & O(n) & O(n) & O(1) & O(1) & O(1) & O(1) \\
		\hline
		queue & O(1) amortized & O(1) & O(n) & O(1) & O(1) & O(1) & O(1) \\
		\hline
		priority queue & O(log n) & O(log n) & O(n) & O(1) & O(1) & O(1) & ??? \\
		\hline
		set & O(log n) & O(log n) & O(log n) & O(1) & O(1) & O(1) & O(1) \\
		\hline
		multi-set & O(log n) & ??? & O(log n) & O(1) & O(1) & O(1) & O(1) \\
		\hline
		map & O(log n) & O(log n) & O(log n) & O(1) & O(1) & O(1) & O(1) \\
		\hline
		multi-map & O(log n) & ??? & O(log n) & O(1) & O(1) & O(1) & O(1) \\
		\hline
		stack & O(1) & O(1) & O(n) & O(1) & O(1) & O(1) & O(1) \\
		\hline
		\end{tabular}
	\end{center}
\end{table}


To conclude, we can get some facts about each data structure: \vspace{-0.3cm}
\begin{enumerate} \itemsep -4pt
\item {\tt std::list} is very very slow to iterate through the collection due to its very poor spatial locality.
\item {\tt std::vector} and {\tt std::deque} perform always faster than {\tt std::list} with very small data
\item {\tt std::list} handles very well large elements
\item {\tt std::deque} performs better than a {\tt std::vector} for inserting at random positions (especially at the front, which is constant time)
\item {\tt std::deque} and {\tt std::vector} do not support very well data types with high cost of copy/assignment
\end{enumerate}



This draws simple conclusions on the usage of each data structure \cite{Bulka2000,Josuttis1999a}: \vspace{-0.3cm}
\begin{enumerate} \itemsep -4pt
\item Number crunching: use std::vector or std::deque
\item Linear search: use std::vector or std::deque
\item Random Insert/Remove:
\item Small data size: use std::vector
\item Large element size: use std::list (unless if intended principally for searching)
\item Non-trivial data type: use std::list unless you need the container especially for searching. But for multiple modifications of the container, it will be very slow.
\item Push to front: use std::deque or std::list
\end{enumerate}






%%%%%%%%%%%%%%%%%%%%%%%%%%%%%%%%%%%%%%%%%%%
\section{Notes About C++}
\label{sec:NotesAboutCpp}


Static variables: \vspace{-0.3cm}
\begin{enumerate} \itemsep -4pt
\item K. Hong, ``C++ Tutorial
Private Inheritance - 2015,'' San Francisco, CA. Available online from {\it Open Source \dots: Java/C++/Python/Android/Design Patterns: C++ Tutorial Home -- 2015} at: \url{}; last accessed on October 23, 2015.
\item K. Hong, ``Static Variables and Static Class Members - 2015,'' San Francisco, CA. Available online from {\it Open Source \dots: Java/C++/Python/Android/Design Patterns: C++ Tutorial Home -- 2015} at: \url{http://www.bogotobogo.com/cplusplus/statics.php}; last accessed on October 23, 2015.
\end{enumerate}












