%	This is written by Zhiyang Ong to document numerical methods that I have implemented for my C++ -based boilerplate code repository.

%	The MIT License (MIT)

%	Copyright (c) <2014> <Zhiyang Ong>

%	Permission is hereby granted, free of charge, to any person obtaining a copy of this software and associated documentation files (the "Software"), to deal in the Software without restriction, including without limitation the rights to use, copy, modify, merge, publish, distribute, sublicense, and/or sell copies of the Software, and to permit persons to whom the Software is furnished to do so, subject to the following conditions:

%	The above copyright notice and this permission notice shall be included in all copies or substantial portions of the Software.

%	THE SOFTWARE IS PROVIDED "AS IS", WITHOUT WARRANTY OF ANY KIND, EXPRESS OR IMPLIED, INCLUDING BUT NOT LIMITED TO THE WARRANTIES OF MERCHANTABILITY, FITNESS FOR A PARTICULAR PURPOSE AND NONINFRINGEMENT. IN NO EVENT SHALL THE AUTHORS OR COPYRIGHT HOLDERS BE LIABLE FOR ANY CLAIM, DAMAGES OR OTHER LIABILITY, WHETHER IN AN ACTION OF CONTRACT, TORT OR OTHERWISE, ARISING FROM, OUT OF OR IN CONNECTION WITH THE SOFTWARE OR THE USE OR OTHER DEALINGS IN THE SOFTWARE.

%	Email address: echo "cukj -wb- 23wU4X5M589 TROJANS cqkH wiuz2y 0f Mw Stanford" | awk '{ sub("23wU4X5M589","F.d_c_b. ") sub("Stanford","d0mA1n"); print $5, $2, $8; for (i=1; i<=1; i++) print "6\b"; print $9, $7, $6 }' | sed y/kqcbuHwM62z/gnotrzadqmC/ | tr 'q' ' ' | tr -d [:cntrl:] | tr -d 'ir' | tr y "\n"

%%%%%%%%%%%%%%%%%%%%%%%%%%%%%%%%%%%%%%%%%%%%%%



%%%%%%%%%%%%%%%%%%%%%%%%%%%%%%%%%%%%%%%%%%%
\chapter{Optimization}
\label{chp:Optimization}





%%%%%%%%%%%%%%%%%%%%%%%%%%%%%%%%%%%%%%%%%%%
\section{Robust Linear Programming}
\label{sec:RobustLinearProgramming}


During the ``lab meeting'' on Friday, December 4, 2015, Prof. Jiang Hu told me that I can transform a robust linear programming into a standard/``standard'' linear programming problem. He told me to look at the references in 

Some of the mathematical programming solvers, including linear programming solvers, are: \vspace{-0.3cm}
\begin{enumerate} \itemsep -4pt
\item {\it LocalSolver} \cite{Innovation24Staff2015}: \vspace{-0.3cm}
	\begin{enumerate} \itemsep -2pt
	\item Hybrid solver for optimization problems
	\item ``Next-generation, hybrid mathematical programming solver'' \cite[Product: Overview]{Innovation24Staff2015}
	\item From \cite[Support Center: Example tour]{Innovation24Staff2015}, {\it LocalSolver} can solve continuous and discrete/combinatorial optimization problems.
	\end{enumerate}
\end{enumerate}



%%%%%%%%%%%%%%%%%%%%%%%%%%%%%%%%%%%%%%%%%%%
\section{Discrete Optimization}
\label{sec:DiscreteOptimization}

Discrete optimization is classified into the following categories \cite{WikipediaContributors2015h,Hammer1979,Lee2004c}: \vspace{-0.3cm}
\begin{enumerate} \itemsep -4pt
\item combinatorial optimization
\item integer programming
\end{enumerate}






















































